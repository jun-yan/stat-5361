\documentclass[twocolumn]{article}
\usepackage[hmargin={0.5in, 0.5in}, vmargin={0.8in, 0.8in}]{geometry}
\usepackage{booktabs}
\usepackage{hyperref}
\usepackage[anythingbreaks]{breakurl}


\usepackage{enumitem}
% \setlist{nolistsep}

\usepackage{fancyhdr}
\pagestyle{fancy}
\renewcommand{\headrulewidth}{0.4pt}
\renewcommand{\footrulewidth}{0.4pt}

\lhead{\sf STAT 5361}
\rhead{Fall 2018}
\lfoot{\sf jun.yan@uconn.edu}
\rfoot{\url{http://www.stat.uconn.edu/~jyan/}}


\begin{document}
% \maketitle


%% to save space; otherwise, would have used \maketitle
\twocolumn[%
\centerline{\Large \bf 
  STAT 5361: Statistical Computing, Fall 2018}
 \medskip
 \centerline{\bf August 27, 2018}
 \bigskip
 ]

\thispagestyle{fancy}


\begin{description}
\item[Instructor:] Jun Yan\\ 
  Department of Statistics\\
  Austin 328\\
  860/486-3416\\
  jun.yan@uconn.edu

\item[Lectures:] 
  TuTh 11:00am -- 12:15 @  GENT 430

\item[Office Hours:] 
  TuTh: 12:30pm -- 1:30pm


\item[Course Description:] From Graduate Catalog:
\begin{quote}
Use of computing for statistical problems; obtaining features of
distributions, fitting models and implementing inference. Basic numerical
methods, nonlinear statistical methods, numerical integration, modern
simulation methods.
\end{quote}

\item[Prerequisite:]
  STAT 3025Q, 3445 or 5685 and/or consent of instructor.  Please check
  the Department's course web page for more information.
 

\item[Textbook:] 
Most materials will come from 
\begin{itemize}
\item Geof H. Givens and Jennifer A. Hoeting,
  \emph{Computational Statistics\/}, Second Edition, John
  Wiley and Sons, 2013.
\end{itemize}

Some materials will come from
\begin{itemize}
\item Paul Glasserman, \emph{Monte Carlo Methods in
      Financial Engineering\/}, Springer, 2004.
\end{itemize}

\item[Course Material:]
Topics include
\begin{enumerate}[noitemsep]
\item Optimization
% , including nonstochastic methods, simulated
%   annealing, genetic algorithms
\item EM algorithms
\item Random sampling
\item Monte Carlo integration and variance reductions, with financial applications
\item Other computing topics of interest
\end{enumerate}

Announcements, lecture notes, homework assignment, and other course
information will be posted on HuskyCT (\url{lms.uconn.edu}). 
You are expected to read the sections of the textbook that will be covered.

\item[Programming Language]
Students are required to use \texttt{RMarkdown}/\texttt{bookdown} and
\texttt{GitHub} for homework assignments, exam, and project.

% Other programming languages or software, such as SAS,
% Excel, C, Fortran, are \ink{0,0,1}{\emph{not\/}} allowed.

A quick introduction to R programming language at
\url{https://cran.r-project.org/doc/manuals/r-release/R-intro.pdf}

For an even shorter introduction see
\url{https://cran.r-project.org/doc/contrib/Torfs+Brauer-Short-R-Intro.pdf}

Write cool homework or project report with \texttt{bookdown} of Yihui Xie:
\url{https://bookdown.org/yihui/bookdown/}

% Learn \texttt{RMarkdown} at
% \url{rmarkdown.rstudio.com/}

Happy git with R of Jenny Bryan:
\url{http://happygitwithr.com/}

Build your own website with \texttt{blogdown}:
\url{https://bookdown.org/yihui/blogdown/}


\item[Grading:] Assessment of the subject will be based on 
the following four components with weights shown in parenthesis:

\begin{description}
\item[Homework (40\%)]
Collaborations are allowed for homework assignments, but each group
should have at most 2 members. No collaborations between groups are
allowed. On the front page of solutions, the members of a group must be
clearly listed. Late homework will \emph{not} be accepted for \emph{any}
reason. 

\item[Exam (30\%)]
There will be one midterm take-home exam. The rule on collaborations for the
midterm exam is the same as for homework assignments.

\item[Project (30\%)]
Each group is expected to complete a class project on a topic of your choice
about statistical computation. There are many possibilities. For example, you
may review an important topic about statistical computation, you may solve a
real problem using comprehensive computation methods, you may investigate
properties of a computational algorithm/strategy, and you may compare several
computation methods on a variety of problems.

A final paper/report of your project using my template should be 6~pages in
length, excluding any appendices you wish to attach. In any case, the project
should present new work, not something you have done for another course.
If you use any reference, you must cite and credit your sources.
Outstanding report will be recommended for publication at the project showcases
of the Data Science Lab (\url{https://statds.org}).

% Your project
% will be shared with the class through the class website on HuskyCT.

\end{description}

% Grading scale will be close to (but may be slightly adjusted): 
% A (87--100), B (73--86), C (60--72), D (50--59), and F (0-50).
% Final grades will include $+/-$.

Grades for the course are assigned totally at the instructor's discretion. A \textbf{rough} guide:
\begin{center}
  \begin{tabular}{lllll}
    A: 91--100\% & A$-$: 89--90\% & B$+$: 87--88\% & B: 81--86\%\\
    B$-$: 79--80\%  & C$+$: 77--78\% & C: 71--76\% &  C$-$: 69--70\%\\
    D$+$: 67--68\% &  D: 61--66\%  & D$-$: 59--60\% & F: $<$ 59\%
  \end{tabular}
\end{center}

\item[Notes:]\hspace{0pt}
\small
  \begin{itemize}
  \item Pick up and sharpen computing skills on the fly.
  \item Work on the problems in the textbook.
  \item International students please consider English as part of your training.
  \item Academic integrity is seriously regarded and academic misconduct
    has severe consequences; students from different cultures beware.
  \item Learn etiquette in academia and practice it when emailing your
    professors.
  \end{itemize}

\end{description}

\end{document}
